\documentclass[a4paper]{article}
\usepackage{fullpage}
\usepackage{enumitem}

\usepackage{graphicx}

\title{ADDIS 1.x architecture description}

\begin{document}

\maketitle

\section{Overview}

ADDIS has a layered architecture based on the `Presentation Model' design pattern.

\begin{figure}[h]
\centering
\includegraphics[scale=0.25]{overview}
\caption{Architecture overview}
\end{figure}

Entities: captures the basic structure of the domain, and guards the integrity of data.
Otherwise, the entities should be relatively simple objects (Java Beans) with few responsibilities.
The entities can be read/written in the ADDIS XML format.
The entities are the in-memory representation of the full data set.
There is no database as such.

Presentation: the presentation layer encapsulates the logic for interacting with the user.
It includes capabilities for formatting and aggregating over properties of entities.
It also has the logic for creating and editing entities.
This serves to make GUI interactions as testable as possible.

GUI: is mainly concerned with the layout and visual presentation of the user interface.
Should defer to the presentation layer for any logic.

ADDIS also integrates the GeMTC and JSMAA components which have a similar architecture.

ADDIS does quite a bit of statistical calculations.
We see these as part of the domain logic.
Therefore if they are reasonably small, such calculations will usually be placed in separate classes in the entities package.
See for example the {\tt entities.relativeeffect} sub-package.
These classes are not serialized to XML like other entities so they are not really entities.
More complex calculations often have their own package, such as lyndobrien and mcmcmodel.

\section{XML serialization}

In ADDIS prior to version 0.8, we used Java object serialization to save entities between sessions.
However, this was very fragile as any minute change to the entities could make loading impossible.
Worse, those errors were non-recoverable because the format is difficult to parse.
Therefore, we switched to an XML format using Javalution.
However, this also had several problems: the XML format was not well-defined (only implicitly in the Java code), the produced XML was not nicely structured, and there was no clear way to maintain backward compatibility when making changes to the data structure.
With ADDIS v1.6 this was solved by the introduction of Schema-based XML serialization.

\begin{figure}[h]
\centering
\includegraphics[scale=0.25]{xml}
\caption{XML serialization and transformation}
\label{fig:xml}
\end{figure}

The XML data structure is normatively defined by the {\tt addis-$n$.xsd} schema (where $n$ is the schema version).
Using JAXB, a [de]serializer is generated from the schema.
JAXB reads/writes the XML and generates java objects to represent the XML structure.
Unfortunately these objects are of low quality and can not be used to replace the entities.
Therefore, there is a JAXBConvertor to convert between the JAXB-generated objects and our hand-crafted entities.

To maintain backward compatibility, we define an XSLT (XSL Transform) definition from the previous to the current version.
These can be chained (see Figure~\ref{fig:xml}) to read data from any previous version.
There is also a {\tt transform0-1.xslt} that was painstakingly constructed to convert the legacy XML to something sensible (and schema-compliant).

\section{Presentation model}

For simple entities, there is a single PresentationModel that provides ValueModels to bind properties of the entity in question to GUI components.
More complicated entities are created using Wizards.
These wizards require a lot of logic to properly guide the step-wise creation of the data structure.
In such cases, there is a separate *WizardPresentation that encapsulates the creation-specific logic.

\begin{figure}[h]
\centering
\includegraphics[scale=0.25]{presentation}
\caption{Example presentation model implementation}
\end{figure}

For example, the NetworkMetaAnalysisWizard has a step where studies can be included or excluded.
It consists of a table listing all available studies and a few of their characteristics.
In the table, studies can be included or excluded using checkboxes.
Steps further on in the wizard use the list of included studies.
The table is built using a TableModel that encodes most of the logic and has tests.
A JTable displays this model.
Each checkbox is backed by a ValueModel that holds either true or false.
The presentation is responsible for appropriately handling changes in the value - the GUI is only concerned with the ValueModel.
The presentation model also provides an ObservableList of selected studies that can be displayed in other wizard steps, or used to build other TableModels, ValueModels, etc.

\section{Computationally intensive tasks}

Some tasks performed by ADDIS are computationally intensive and inherently parallel.
To support such tasks, the {\tt org.drugis.common.threading} package has a flexible task scheduler that can handle tasks composed of (parallel or sequential) subtasks.
It will automatically run as many parallel tasks as there are CPU cores available.

\begin{figure}[h]
\centering
\includegraphics[scale=0.25]{activitytask}
\caption{The ActivityTask corresponding to a network meta-analysis with two parallel chains}
\end{figure}

Such complex tasks are modeled as {\tt ActivityTasks}, using the language of UML activity diagrams.
An example activity is shown for a network meta-analysis with 2 chains.

Any computation that takes more than a few seconds should be scheduled using this mechanism.

\end{document}
